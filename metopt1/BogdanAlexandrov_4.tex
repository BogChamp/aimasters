\documentclass[12pt]{extreport}
\usepackage[T2A]{fontenc}
\usepackage[utf8]{inputenc}        % Кодировка входного документа;
                                    % при необходимости, вместо cp1251
                                    % можно указать cp866 (Alt-кодировка
                                    % DOS) или koi8-r.

\usepackage[english,russian]{babel} % Включение русификации, русских и
                                    % английских стилей и переносов
%%\usepackage{a4}
%%\usepackage{moreverb}
\usepackage{amsmath,amsfonts,amsthm,amssymb,amsbsy,amstext,amscd,amsxtra,multicol}
\usepackage{indentfirst}
\usepackage{verbatim}
\usepackage{tikz} %Рисование автоматов
\usetikzlibrary{automata,positioning}
\usepackage{multicol} %Несколько колонок
\usepackage{graphicx}
\usepackage[colorlinks,urlcolor=blue]{hyperref}
\usepackage[stable]{footmisc}


%% \voffset-5mm
%% \def\baselinestretch{1.44}
\renewcommand{\theequation}{\arabic{equation}}
\def\hm#1{#1\nobreak\discretionary{}{\hbox{$#1$}}{}}
\newtheorem{Lemma}{Лемма}
\theoremstyle{definiton}
\newtheorem{Remark}{Замечание}
%%\newtheorem{Def}{Определение}
\newtheorem{Claim}{Утверждение}
\newtheorem{Cor}{Следствие}
\newtheorem{Theorem}{Теорема}
\theoremstyle{definition}
\newtheorem{Example}{Пример}
\newtheorem*{known}{Теорема}
\def\proofname{Доказательство}
\theoremstyle{definition}
\newtheorem{Def}{Определение}

%% \newenvironment{Example} % имя окружения
%% {\par\noindent{\bf Пример.}} % команды для \begin
%% {\hfill$\scriptstyle\qed$} % команды для \end






%\date{22 июня 2011 г.}
\let\leq\leqslant
\let\geq\geqslant
\def\MT{\mathrm{MT}}
%Обозначения ``ажуром''
\def\BB{\mathbb B}
\def\CC{\mathbb C}
\def\RR{\mathbb R}
\def\SS{\mathbb S}
\def\ZZ{\mathbb Z}
\def\NN{\mathbb N}
\def\FF{\mathbb F}
%греческие буквы
\let\epsilon\varepsilon
\let\es\varnothing
\let\eps\varepsilon
\let\al\alpha
\let\sg\sigma
\let\ga\gamma
\let\ph\varphi
\let\om\omega
\let\ld\lambda
\let\Ld\Lambda
\let\vk\varkappa
\let\Om\Omega
\def\abstractname{}

\def\R{{\cal R}}
\def\A{{\cal A}}
\def\B{{\cal B}}
\def\C{{\cal C}}
\def\D{{\cal D}}

%классы сложности
\def\REG{{\mathsf{REG}}}
\def\CFL{{\mathsf{CFL}}}


%%%%%%%%%%%%%%%%%%%%%%%%%%%%%%% Problems macros  %%%%%%%%%%%%%%%%%%%%%%%%%%%%%%%


%%%%%%%%%%%%%%%%%%%%%%%% Enumerations %%%%%%%%%%%%%%%%%%%%%%%%

\newcommand{\Rnum}[1]{\expandafter{\romannumeral #1\relax}}
\newcommand{\RNum}[1]{\uppercase\expandafter{\romannumeral #1\relax}}

%%%%%%%%%%%%%%%%%%%%% EOF Enumerations %%%%%%%%%%%%%%%%%%%%%

\usepackage{xparse}
\usepackage{ifthen}
\usepackage{bm} %%% bf in math mode
\usepackage{color}
%\usepackage[usenames,dvipsnames]{xcolor}

\definecolor{Gray555}{HTML}{555555}
\definecolor{Gray444}{HTML}{444444}
\definecolor{Gray333}{HTML}{333333}


\newcounter{problem}
\newcounter{uproblem}
\newcounter{subproblem}
\newcounter{prvar}

\def\beforPRskip{
	\bigskip
	%\vspace*{2ex}
}

\def\PRSUBskip{
	\medskip
}


\def\pr{\beforPRskip\noindent\stepcounter{problem}{\bf \theproblem .\;}\setcounter{subproblem}{0}}
\def\pru{\beforPRskip\noindent\stepcounter{problem}{\bf $\mathbf{\theproblem}^\circ$\!\!.\;}\setcounter{subproblem}{0}}
\def\prstar{\beforPRskip\noindent\stepcounter{problem}{\bf $\mathbf{\theproblem}^*$\negthickspace.}\setcounter{subproblem}{0}\;}
\def\prpfrom[#1]{\beforPRskip\noindent\stepcounter{problem}{\bf Задача \theproblem~(№#1 из задания).  }\setcounter{subproblem}{0} }
\def\prp{\beforPRskip\noindent\stepcounter{problem}{\bf Задача \theproblem .  }\setcounter{subproblem}{0} }

\def\prpvar{\beforPRskip\noindent\stepcounter{problem}\setcounter{prvar}{1}{\bf Задача \theproblem \;$\langle${\rm\Rnum{\theprvar}}$\rangle$.}\setcounter{subproblem}{0}\;}
\def\prpv{\beforPRskip\noindent\stepcounter{prvar}{\bf Задача \theproblem \,$\bm\langle$\bracketspace{{\rm\Rnum{\theprvar}}}$\bm\rangle$.  }\setcounter{subproblem}{0} }
\def\prv{\beforPRskip\noindent\stepcounter{prvar}{\bf \theproblem\,$\bm\langle$\bracketspace{{\rm\Rnum{\theprvar}}}$\bm\rangle$}.\setcounter{subproblem}{0} }

\def\prpstar{\beforPRskip\noindent\stepcounter{problem}{\bf Задача $\bf\theproblem^*$\negthickspace.  }\setcounter{subproblem}{0} }
\def\prdag{\beforPRskip\noindent\stepcounter{problem}{\bf Задача $\theproblem^{^\dagger}$\negthickspace\,.  }\setcounter{subproblem}{0} }
\def\upr{\beforPRskip\noindent\stepcounter{uproblem}{\bf Упражнение \theuproblem .  }\setcounter{subproblem}{0} }
%\def\prp{\vspace{5pt}\stepcounter{problem}{\bf Задача \theproblem .  } }
%\def\prs{\vspace{5pt}\stepcounter{problem}{\bf \theproblem .*   }
\def\prsub{\PRSUBskip\noindent\stepcounter{subproblem}{\sf \thesubproblem .} }
\def\prsubr{\PRSUBskip\noindent\stepcounter{subproblem}{\bf \asbuk{subproblem})}\;}
\def\prsubstar{\PRSUBskip\noindent\stepcounter{subproblem}{\rm $\thesubproblem^*$\negthickspace.  } }
\def\prsubrstar{\PRSUBskip\noindent\stepcounter{subproblem}{$\text{\bf \asbuk{subproblem}}^*\mathbf{)}$}\;}

\newcommand{\bracketspace}[1]{\phantom{(}\!\!{#1}\!\!\phantom{)}}

\DeclareDocumentCommand{\Prpvar}{ O{null} O{} }{
	\beforPRskip\noindent\stepcounter{problem}\setcounter{prvar}{1}{\bf Задача \theproblem
% 	\ifthenelse{\equal{#1}{null}}{  }{ {\sf $\bm\langle$\bracketspace{#1}$\bm\rangle$}}
%	~\!\!(\bracketspace{{\rm\Rnum{\theprvar}}}).  }\setcounter{subproblem}{0}
%	\;(\bracketspace{{\rm\Rnum{\theprvar}}})}\setcounter{subproblem}{0}
%
	\,{\sf $\bm\langle$\bracketspace{{\rm\Rnum{\theprvar}}}$\bm\rangle$}
	~\!\!\! \ifthenelse{\equal{#1}{null}}{\!}{{\sf(\bracketspace{#1})}}}.

}
%\DeclareDocumentCommand{\Prpvar}{ O{level} O{meta} m }{\prpvar}


\DeclareDocumentCommand{\Prp}{ O{null} O{null} }{\setcounter{subproblem}{0}
	\beforPRskip\noindent\stepcounter{problem}\setcounter{prvar}{0}{\bf Задача \theproblem
	~\!\!\! \ifthenelse{\equal{#1}{null}}{\!}{{\sf(\bracketspace{#1})}}
	 \ifthenelse{\equal{#2}{null}}{\!\!}{{\sf [\color{Gray444}\,\bracketspace{{\fontfamily{afd}\selectfont#2}}\,]}}}.}

\DeclareDocumentCommand{\Pr}{ O{null} O{null} }{\setcounter{subproblem}{0}
	\beforPRskip\noindent\stepcounter{problem}\setcounter{prvar}{0}{\bf\theproblem
	~\!\!\! \ifthenelse{\equal{#1}{null}}{\!\!}{{\sf(\bracketspace{#1})}}
	 \ifthenelse{\equal{#2}{null}}{\!\!}{{\sf [\color{Gray444}\,\bracketspace{{\fontfamily{afd}\selectfont#2}}\,]}}}.}

%\DeclareDocumentCommand{\Prp}{ O{level} O{meta} }

\DeclareDocumentCommand{\Prps}{ O{null} O{null} }{\setcounter{subproblem}{0}
	\beforPRskip\noindent\stepcounter{problem}\setcounter{prvar}{0}{\bf Задача $\bm\theproblem^* $
	~\!\!\! \ifthenelse{\equal{#1}{null}}{\!}{{\sf(\bracketspace{#1})}}
	 \ifthenelse{\equal{#2}{null}}{\!\!}{{\sf [\color{Gray444}\,\bracketspace{{\fontfamily{afd}\selectfont#2}}\,]}}}.
}

\DeclareDocumentCommand{\Prpd}{ O{null} O{null} }{\setcounter{subproblem}{0}
	\beforPRskip\noindent\stepcounter{problem}\setcounter{prvar}{0}{\bf Задача $\bm\theproblem^\dagger$
	~\!\!\! \ifthenelse{\equal{#1}{null}}{\!}{{\sf(\bracketspace{#1})}}
	 \ifthenelse{\equal{#2}{null}}{\!\!}{{\sf [\color{Gray444}\,\bracketspace{{\fontfamily{afd}\selectfont#2}}\,]}}}.
}

\DeclareDocumentCommand{\PR}{ O{null} }{\setcounter{subproblem}{0}
	\medskip\noindent\stepcounter{problem}\setcounter{prvar}{0}{\bf\theproblem~{\sf(\bracketspace{#1})}.}}	



\def\prend{
	\medskip
%	\bigskip
%	\bigskip
}




%%%%%%%%%%%%%%%%%%%%%%%%%%%%%%% EOF Problems macros  %%%%%%%%%%%%%%%%%%%%%%%%%%%%%%%



%\usepackage{erewhon}
%\usepackage{heuristica}
%\usepackage{gentium}

\usepackage[portrait, top=3cm, bottom=1.5cm, left=3cm, right=2cm]{geometry}

\usepackage{fancyhdr}
\pagestyle{fancy}
\renewcommand{\headrulewidth}{0pt}
\lhead{\fontfamily{fca}\selectfont {AI Masters :: МетОпты 2022} }
%\lhead{ \bf  {ТРЯП. } Семинар 1 }
%\chead{\fontfamily{fca}\selectfont {Вариант 1}}
\rhead{\fontfamily{fca}\selectfont Домашнее задание 4}
%\rhead{\small 01.09.2016}
\cfoot{}

\usepackage{titlesec}
\titleformat{\section}[block]{\Large\bfseries\filcenter {\setcounter{problem}{0}}  }{}{1em}{}


%%%%%%%%%%%%%%%%%%%%%%%%%%%%%%%%%%%%%%%%%%%%%%%%%%%% Обозначения и операции %%%%%%%%%%%%%%%%%%%%%%%%%%%%%%%%%%%%%%%%%%%%%%%%%%%% 
                                                                    
\newcommand{\divisible}{\mathop{\raisebox{-2pt}{\vdots}}}           
\let\Om\Omega


%%%%%%%%%%%%%%%%%%%%%%%%%%%%%%%%%%%%%%%% Shen Macroses %%%%%%%%%%%%%%%%%%%%%%%%%%%%%%%%%%%%%%%%
\newcommand{\w}[1]{{\hbox{\texttt{#1}}}}

\usepackage{enumitem}
\usepackage[linesnumbered]{algorithm2e}   

\newcommand{\comments}[2][Комментарий]{
\medskip
	\noindent{\bfseries #1: }{\textsl{#2}}
%\medskip	
}

\def\ukaz{\noindent\textbf{Указание.} }

\begin{document}
	\PR[] Условия оптимальности
    \begin{enumerate} 
        \item Построим лагранжиан:
        \\$L(x,\mu) = c^Tx + \mu(x^TAx-1)$
        \\ Составим условия ККТ для решения задачи:
        \begin{enumerate}
            \item $L_x^{'}=c+2\mu Ax=0$
            \item $\mu(x^TAx-1)=0$
            \item $\mu \geq 0$
        \end{enumerate}
        \\ Заметим, что $\mu \neq 0$, иначе из первого условия вытекает $c=0$, что противоречит условию.(минимизировать 0 не имеет смысла.)
        \\Таким образом, должно выполняться равенство $x^TAx=1$. Выразим из первого условия $x=-\frac{1}{2\mu}A^{-1}c$. Подставим в полученное условие:
        \\$\frac{1}{2\mu}c^TA^{-T}A\frac{1}{2\mu}A^{-1}c=\frac{1}{4\mu^2}c^TA^{-1}c=1$, $A=A^T$ из условия.
        \\Таким образом, $\mu = \sqrt{c^TA^{-1}c}/2$ и $x^* = -\frac{A^{-1}c}{\sqrt{c^TA^{-1}c}}$.
        \\ Минимум таким образом будет равен $c^T*x^*=-\frac{c^TA^{-1}c}{\sqrt{c^TA^{-1}c}}=-\sqrt{c^TA^{-1}c}$
        \item Преобразуем сначала условие $Xz=y \to z^TXz=1$, так как $z^Ty=1$.
        \\$L(x,\mu) = trace(X)-\log(\det X) + \mu(z^TXz-1)$
        \\ Условия ККТ:
        \begin{enumerate}
            \item $L_x^{'}=I-X^{-1}+\mu zz^T=0$, так как $X \in S_{++}^n$
            \item $z^TXz=1$
        \end{enumerate}
        \\Из первого уравнения следует $X = (I + \mu zz^T)^{-1}$. Подставим этот результат во второе: $z^T(I + \mu zz^T)^{-1}z=1=y^Tz \to (I + \mu zz^T)^{-1}z=y \to z = (I + \mu zz^T)y = y + \mu z \to z(1-\mu) = y \to 1 -\mu = y^Ty \to \mu = 1 - y^Ty$
        \\ $X = (I+(1 - y^Ty)zz^T)^{-1}$
        \\ Минимум будет $trace((I+(1 - y^Ty)zz^T)^{-1})-\log(\det (I+(1 - y^Ty)zz^T)^{-1})$
        \item Поставим формально задачу:
        \\$min\ (Ax)^TAx$, $x^T1=1$, $x \geq 0$, где A - матрица размером n*k, ее столбцы - вектора $a_i$. Мы минимизируем квадрат нормы вектора из выпуклой оболочки, так как минимум будет в той же точке, что и для минимума самой нормы вектора.
        \\$L(x, \lambda, \mu)= (Ax)^TAx + \lambda(x^T1 - 1)-\mu^Tx$
        \begin{enumerate}
            \item $L_x^{'}=2A^TAx - \lambda1-\mu = 0$
            \item $\mu \geq 0$
            \item $\mu^Tx = 0$
            \item $x^T1=1$
            \item $x \geq 0$
        \end{enumerate}
        \\$\mu = 2A^TAx - \lambda1$ из первого условия.
        \\$x^T\mu = 2x^TA^TAx - \lambda x^T1=2x^TA^TAx - \lambda=0 \to \lambda = 2x^TA^TAx$. Таким образом, $\mu = 2A^TAx - 2x^TA^TAx*1$
        \\ Остается решить систему $2A^TAx - 2x^TA^TAx*1 \geq 0$, $x \geq 0$. Пусть ее решение - $x*$. Тогда минимум задачи будет $(Ax^*)^TAx^*$, а значит минимальная норма - $\sqrt{(Ax^*)^TAx^*}$
        \item $L(x, \lambda, \mu) = -\sum_i^n \log(\alpha_i+x_i)+\lambda(1^Tx-1)-\mu^Tx$
        \\ Функция является дважды дифференциируемой, значит мы можем попробовать найти L > 0, что функция будет L-гладкой.
        \\ $L_{x_k}^{'} = -1/(\alpha_k+x_k)+\lambda-\mu_k$, $L_\lambda^{'} = 1^Tx - 1$, $L_{\mu_k}^{'} = -x_k$
        \\ $L_{x_kx_k}^{''} =1/(\alpha_k+x_k)^2$, $L_{x_kx_i}^{''} =0$, $L_{\lambda \lambda}^{''} = 0$, $L_{\mu \mu}^{''} = 0$, $L_{x \lambda}^{''} = 1$, $L_{x \mu}^{''} = -I$, $L_{\lambda \mu}^{''} = 0$
        \\ Гессиан выглядит как $L^{''} = \begin{pmatrix}
        A & 1 & -I\\
        1^T & 0 & 0^T\\
        -I & 0 & 0
        \end{pmatrix}$, где $A=diag(1/(\alpha_k+x_k)^2)$. Если мы будем искать собственные значения этого гессиана, то по формуле фробениуса $det(L^{''}-aI) = det(A-aI)det(-aI-\begin{pmatrix}
        1^T\\
        -I\end{pmatrix}A^{-1}\begin{pmatrix}
        1 (-I) \end{pmatrix}) = 0$. Но у матрицы $det(\begin{pmatrix}
        1^T\\
        -I\end{pmatrix}A^{-1}\begin{pmatrix}
        1 (-I) \end{pmatrix})$ определитель равен нулю, значит есть собственное значение равно нулю, значит и у исходной матрицы есть собственное значение равное нулю. Соотвественно L не сильно выпукла.
        \\ Соответственно, у нас просто выпуклая функция, и если применить метод градиентного спуска, то по теореме скорость будет сублинейная.
        \\ $(x_{k+1}, \lambda_{k+1}, \mu_{k+1})=(x_{k}, \lambda_{k}, \mu_{k})-\alpha (L_x^{'}(x_{k}, \lambda_{k}, \mu_{k}), 1^Tx_k - 1, -x_k)$, где $\alpha \leq 1/(max\_eigenvalue)$.
        \item $L(x, \lambda, \mu) = \|x-y\|_2^2+\lambda(1^Tx-1)-\mu^Tx$
        \\ Функция является дважды дифференциируемой, значит мы можем попробовать найти L > 0, что функция будет L-гладкой.
        \\ $L_x^{'} = 2(x-y)+\lambda 1 - \mu$, $L_\lambda^{'} = 1^Tx - 1$, $L_\mu^{'} = -x$
        \\ $L_{xx}^{''} = 2$, $L_{\lambda \lambda}^{''} = 0$, $L_{\mu \mu}^{''} = 0$, $L_{x \lambda}^{''} = 1$, $L_{x \mu}^{''} = -I$, $L_{\lambda \mu}^{''} = 0$
        \\ Гессиан выглядит как $L^{''} = \begin{pmatrix}
        2 & 1 & -I\\
        1^T & 0 & 0^T\\
        -I & 0 & 0
        \end{pmatrix}$, размерности (2*n+1) на (2*n+1), где n - размерность x. Собственные значения этой матрицы будут такие: 0 порядка 1, 1 порядка n-1, -1 порядка n-1, $-\sqrt{n*(n+1)+1}+n$ и $\sqrt{n*(n+1)+1}+n$.
        \\ $\|L^{''}\|_2 = \sqrt{n*(n+1)+1}+n$, поэтому можем взять $L = \sqrt{n*(n+1)+1}+n$. Так как у гессиана есть собственное значение 0, то функция не может быть сильно выпуклой.
        \\ Рассмотрим какую-то начальную точку $x_0 \geq 0, 1^Tx_0=1, \lambda_0, \mu_0 \geq 0$ и итерационный процесс $(x_{k+1}, \lambda_{k+1}, \mu_{k+1})=(x_{k}, \lambda_{k}, \mu_{k})-\alpha (2(x_k - y) + \lambda_k 1 - \mu_k, 1^Tx_k - 1, -x_k)$ с $\alpha \leq 1/L = 1/(\sqrt{n*(n+1)+1}+n)$. По теореме о градиентом спуске $L_k-L^* \leq \frac{\|(x^*, \lambda^*, \mu^*)-(x_0,\lambda_0, \mu_0)\|_2^2}{2\alpha k} = O(1/k)$ - сублинейная скорость сходимости.
    \end{enumerate}
    \prend

    \PR[] Двойственные задачи
    \begin{enumerate}
    \item $L(x, \lambda, \mu) = c^Tx + \sum_{i=1}^n\lambda_ix_i(1-x_i)+\mu^T(Ax-b)=c^Tx+\lambda^Tx-x^Tdiag(\lambda)x+\mu^T Ax-\mu^T b=(c+\lambda+ A^T\mu)^Tx-x^Tdiag(\lambda)x-\mu^T b$
    \\ $g(\lambda, \mu) = inf_x((c+\lambda+ A^T\mu)^Tx-x^Tdiag(\lambda)x-\mu^T b)$
    \\ Заметим, что $diag(\lambda) \succ 0$, иначе можно будет подобрать такой х, что $g \to \infty$.
    \\ Рассмотрим $(c_i + \lambda_i+(A^T)_i\mu)x_i-\lambda_ix_i^2$, где $(A^T)_i$ - i-ая строка матрицы. Минимум данного выражения достигается в точке $x_i^*=(c_i + \lambda_i+(A^T)_i\mu)/(2\lambda_i)$, и минимум равен $(c_i + \lambda_i+(A^T)_i\mu)^2/(4\lambda_i)$
    \\ Отсюда $g(\lambda, \mu)=1/4 *(c+\lambda+ A^T\mu)^Tdiag(\lambda)^{-1}(c+\lambda+ A^T\mu)-\mu^Tb$
    \\ Таким образом, двойственная задача:
    \\ $$max_{\lambda, \mu} 1/4 *(c+\lambda+ A^T\mu)^Tdiag(\lambda)^{-1}(c+\lambda+ A^T\mu)-\mu^Tb$$\\ s.t $\lambda > 0,\ \mu \geq 0$.
    \\ Найдем двойственную задачу к задаче непрерывной релаксации:
    \\ $L_1(x,\mu_1, \mu_2, \mu_3) = c^Tx - \mu_1^Tx+\mu_2^T(x-1)+\mu_3^T(Ax-b) = (c-\mu_1+\mu_2+A^T\mu_3)^Tx-\mu_2^T1-\mu_3^Tb$ - лагранжиан задачи непрерывной релаксации.
    \\ $$max\ -\mu_2^T1-\mu_3^Tb$$ $$c-\mu_1+\mu_2+A^T\mu_3 = 0$$
    \\ Вернемся к двойственной исходной задаче. Заметим, что $sup((c_i + \lambda_i+(A^T)_i\mu)^2/(4\lambda_i)) = c_i+(A^T)_i\mu$, если $c_i+(A^T)_i\mu \leq 0$, иначе 0.
    \\ Отсюда, можно переписать задачу, как $$-\mu^Tb+\sum_{i=1}^n min(0, c_i+(A^T)_i\mu)$$ $$\mu \geq 0$$
    \\ Видно, что данная задача эквивалентна двойственной задаче непрерывной релаксации, соответсвенно нижняя оценка будет такой же.
    \newline
    \\ Если посмотреть на исходную булеву задачу и на ее непрерывную релаксацию, то понятно, что задача непрерывной релаксации дает большее пространство поиска решений, так как допустимое множество исходной задачи - вершины булева куба, а у исходной релаксации - весь булев куб с внутренними точками. 
    \\ Допустим, решение задачи непрерывной релаксации - вектор $x^* = (x_1^*, x_2^*, \ldots, x_n^*)$. Для всех элементов выполняется условие $0 \leq x_i^* \leq 1$. Вектор $x_^*$ можно дополнить до булевого вектора, добавив такой вектор $a$, что $x_i^*+a_i \in \{0,1\}$. Из предположения, что $x_^*$ - решение задачи непрерывной релаксации, следует $f(x^*) = min\ f(x) \leq f(x+a)$. Таким образом, оценка задачи непрерывной релаксации не больше, чем решение исходной булевой задачи.
    \item Исходная задача эквивалентная следующей: $min\ t$ s.t. $a_ix+b_i \leq t$.
    \\$c = (0,1)^T$, $y = (x,t)^T$, $A = 
    \begin{pmatrix}
    a_1^T -1\\
    \ldots\\
    a_m^T -1
    \end{pmatrix}$, $b = (b1,b2, \ldots b_m)^T$
    \\ Таким образом, можно сформулировать задачу как $min_y\ c^Ty$ s.t. $Ay \leq b$.
    \\ $L(x, \mu) = c^Ty + \mu^T(Ay + b) = (c+A^T\mu)^Ty + \mu^Tb$. Так что, $g(\mu)= \mu^Tb$, если $c+A^T\mu = 0$, иначе $\infty$.
    \\ Двойственная задача выглядит как $$max\ \mu^Tb$$ $$\mu \geq 0$$ $$c+A^T\mu = 0$$
    \item 
    $L(X, \mu) = -\log \det X +\sum_{i=1}^m \mu_i(a_i^TXa_i-1) = -\log \det X +\sum_{i=1}^m \mu_ia_i^TXa_i - \mu^T1$. Условие на ограничение можно заменить на $trace(a_ia_i^TX) \leq 1$, эти условия эквивалентны.
    \\ $L_x^{'} = -X^{-1} + \sum_{i=1}^m \mu_i a_ia_i^T = 0 \to X = (\sum_{i=1}^m \mu_i a_ia_i^T)^{-1}$ - это значение минимизирует лагранжиан по первому условию оптимальности.(при условии, что $\sum_{i=1}^m \mu_i a_ia_i^T \succ 0$)
    \\ Значит, $g(\mu) = \log \det (\sum_{i=1}^m \mu_i a_ia_i^T) + trace(\sum_{i=1}^m \mu_ia_ia_i^T(\sum_{i=1}^m \mu_i a_ia_i^T)^{-1}) - \mu^T1 = \log \det (\sum_{i=1}^m \mu_i a_ia_i^T)+n - \mu^T1$
    \\ Двойственная задача будет выглядеть, как $$max \log \det (\sum_{i=1}^m \mu_i a_ia_i^T)+n - \mu^T1$$ $$\mu \geq 0$$ $$\sum_{i=1}^m \mu_i a_ia_i^T \succ 0$$
    \item 
    \begin{enumerate}
        \item $L(x,\lambda) = -3x_1^2+x_2^2+2x_3^2+2(x_1+x_2+x_3)+\lambda(x_1^2+x_2^2+x_3^2-1)$
        \\ Условия ККТ: 
        \begin{enumerate}
            \item $L_{x_1}^{'} = -6x_1+2+2\lambda x_1 = 0 \to x_1 = 1/(3-\lambda)$
            \item $L_{x_2}^{'} = 2x_2+2+2\lambda x_2 = 0 \to x_2 = -1/(1+\lambda)$
            \item $L_{x_3}^{'} = 4x_3+2+2\lambda x_3 = 0 \to x_3 = -1/(2+\lambda)$
            \item $x_1^2+x_2^2+x_3^2 = 1$
        \end{enumerate}
        \\ Выразив $x_1,x_2,x_3$ через $\lambda$, подставим их в условие и получим уравнение. Решив систему, мы получим несколько точек: $\lambda_1 \approx -3,15$, $\lambda_2 \approx 0.22$, $\lambda_3 \approx 1.89$, $\lambda_4 \approx 4.03$.
        \\ Но чтобы точка действительно была минимумом, необходимо, чтобы гессиан в этой точке был положительно определен. В нашем случае, гессиан будет диагональной матрицей, с элементами $-6+2\lambda$, $2+2\lambda$, $4+2\lambda$ на диагонали. Чтобы матрица была положительно опрееделенной, нам по критреию сильвестра, надо потребовать чтобы главные миноры были положительны, что эквивалентно в нашем случае положительности вторых производных. Таким образом, $\lambda > 3$, значит нам подходит одна точка с $\lambda \approx 4.03$.
        \\ $x_1 \approx -0.97$, $x_2 \approx -0.199$, $x_3 \approx 0.166$. Минимум будет равен $\approx -5.4$
        \item Если посмотреть на функцию, которую надо минимизировать, то она содержит как положительные, так и отрицательные квадраты. Квадратичная функция - выпукла, отрицательная к ней - нет, поэтому задача не является выпуклой.
        \item $g(\lambda) = \inf (-3x_1^2+x_2^2+2x_3^2+2(x_1+x_2+x_3)+\lambda(x_1^2+x_2^2+x_3^2-1)) = \frac{-3}{(3-\lambda)^2} + \frac{1}{(1+\lambda)^2} + \frac{2}{(2+\lambda)^2}+2(\frac{1}{3-\lambda} - \frac{1}{1+\lambda} - \frac{1}{2+\lambda})+\lambda(\frac{1}{(3-\lambda)^2} + \frac{1}{(1+\lambda)^2} + \frac{1}{(2+\lambda)^2}) -\lambda = \frac{1}{3-\lambda}-\frac{1}{1+\lambda}-\frac{1}{2+\lambda}-\lambda$, $\lambda > 3$.
        \\ Двойственной задачей будет $$max\ \frac{1}{3-\lambda}-\frac{1}{1+\lambda}-\frac{1}{2+\lambda}-\lambda$$ $$\lambda > 3$$
        \\ Максимум будет достигаться в той же точке $\lambda \approx 4.03$, и значение совпадает с оценкой исходной задачи, так что да, строгая двойственность присутствует.
    \end{enumerate}
    \item Сделаем замену $u_i = A_ix+b_i$
    \\ $L(x, \lambda_1,\lambda_2 \ldots \lambda_p) = 1/2 * \|x-x_0\|_2^2 + \sum_{i=1}^p\|u_i\|_2 + \sum_{i=1}^p\lambda_i^T(u_i - A_ix - b_i) = 1/2 * \|x-x_0\|_2^2 - \sum_{i=1}^p\lambda_i^TA_ix + \sum_{i=1}^p(\|u_i\|_2+\lambda_i^Tu_i) - \sum_{i=1}^p\lambda_i^Tb_i$
    \\ Для каждого i верно, что $inf(\|u_i\|_2+\lambda_i^Tu_i) = 0$, если $\|\lambda_i\|_2 \leq 1$, иначе 0. Здесь применили свойство, что норма l2 сопряженнная сама к себе.
    \\ Чтобы минимизировать по x, найдем градиент $(1/2 * \|x-x_0\|_2^2 - \sum_{i=1}^p\lambda_i^TA_ix)_x^{'}=x - x_0 - \sum_{i=1}^p A_i^T\lambda_i = 0 \to x = x_0 + \sum_{i=1}^p A_i^T\lambda_i$
    \\ Отсюда следует, что $g(\lambda_1,\lambda_2 \ldots \lambda_p) = 1/2*\|\sum_{i=1}^p A_i^T\lambda_i\|_2^2 - \sum_{i=1}^p\lambda_i^TA_ix_0 - \|\sum_{i=1}^p A_i^T\lambda_i\|_2^2 - \sum_{i=1}^p\lambda_i^Tb_i = -1/2*\|\sum_{i=1}^p A_i^T\lambda_i\|_2^2 - \sum_{i=1}^p\lambda_i^T(A_ix_0+b_i)$
    \\ Двойственная задача будет иметь вид: 
    $$max\ -1/2*\|\sum_{i=1}^p A_i^T\lambda_i\|_2^2 - \sum_{i=1}^p\lambda_i^T(A_ix_0+b_i)$$ $$\|\lambda_i\|_2 \leq 1$$
    \end{enumerate}
    \prend


\end{document}
  